%%%%%%%%%%%%%%%%%%%%%%%%%%%%%%%%%%%%%%%%%
%
% (c) 2022 by Jennifer Laaser
%
% This work is licensed under the Creative Commons Attribution-NonCommercial-ShareAlike 4.0 International License. To view a copy of this license, visit http://creativecommons.org/licenses/by-nc-sa/4.0/ or send a letter to Creative Commons, PO Box 1866, Mountain View, CA 94042, USA.
%
% The current source for these materials is accessible on Github: https://github.com/jlaaser/pogil-polymers
%
%%%%%%%%%%%%%%%%%%%%%%%%%%%%%%%%%%%%%%%%%

\renewcommand{\figpath}{content/polymphys/scattering/light-scattering/figs}
\renewcommand{\labelbase}{light-scattering}

\begin{activity}{Static and Dynamic Light Scattering}
\label{\labelbase}

\begin{instructornotes}
	This activity introduces students to concepts related to static and dynamic light scattering from polymer solutions.
	
	After completing this activity, students will be able to:
	\begin{enumerate}
		\item \dots
	\end{enumerate}
	
	\subsection*{Activity summary:}
	\begin{itemize}
		\item \textbf{Activity type:} Learning Cycle
		\item \textbf{Content goals:} See above 
		\item \textbf{Process goals:} %https://pogil.org/uploads/attachments/cj54b5yts006cklx4hh758htf-process-skills-official-pogil-list-2015-original.pdf
			\begin{enumerate}
				\item Interpretation of graphical data
				\item Written and oral communication of reasoning
			\end{enumerate}
		\item \textbf{Duration:} TBD minutes, including time for class discussion
		\item \textbf{Instructor preparation required:} none beyond knowledge of relevant content
		\item \textbf{Related textbook chapters:}
			\begin{itemize}
				\item \emph{Polymer Chemistry} (Hiemenz \& Lodge): section XX
				\item \emph{Introduction to Polymers} (Young \& Lovell): section YY
			\end{itemize}
		%\item \textbf{Instructor notes:}
		%	\begin{itemize}
		%		\item \dots
		%	\end{itemize}
	\end{itemize}
	
\end{instructornotes}



\begin{model}[Static Light Scattering]
	\label{\labelbase:mdl:SLS}
	
	
	
\end{model}


\begin{ctqs}

	\question 

	\question As you learned in Activity \ref{scattering-fundamentals}, the intensity of 
	
\end{ctqs}


\begin{model}[Dynamic Light Scattering]
	\label{\labelbase:mdl:DLS}
	
	\dots
	
\end{model}

\begin{ctqs}

	\question 
	
\end{ctqs}





%\begin{exercises}

%	\exercise 
	
%\end{exercises}


%\begin{problems}
%
%	\problem First exercise
%	\problem Second exercise
%	
%\end{problems}


	
\end{activity}