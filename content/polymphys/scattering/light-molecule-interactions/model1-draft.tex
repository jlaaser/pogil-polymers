%%%%% This model was originally drafted as model 1 of the activity on light scattering.  That activity, however, stands alone without this model, so the draft has been moved to this file as a placeholder for a future Extension activity on interactions of molecules with light.

%\begin{model}[Interaction of Light with Molecules]
%	\label{\labelbase:mdl:lightmolecule}
%	
%	When a molecule is placed into an electric field, its electron cloud distorts (polarizes) to form an induced dipole, as shown below:
%	
%	IMAGE
%	
%	The greater the polarizability of the molecule, $\alpha$, the easier it is for the electric field to move the electrons around, and the larger the induced dipole.
%
%\end{model}
%
%\begin{ctqs}
%
%	\question Sketch the configuration of the electron cloud when the molecule is placed in each of the following fields:
%	
%		IMAGES
%		
%	\question The oscillating field in the previous problem creates an oscillating dipole.  Plot the dipole moment as a function of time on the following axes:
%	
%		AXES
%		
%%	\question Some molecules are easy to polarize (e.g. they don't hold onto their electrons very tightly, so it is easy for the electric field to move them around), while others are hard to polarize.  
%%	
%%		How would the dipole moment you plotted in the previous question differ if the molecule was very hard to polarize?  Sketch your answer on the following axes:
%%		
%%		AXES
%		
%	\question An oscillating dipole creates its \emph{own} oscillating electric field, which can be the source of a new electromagnetic wave (i.e. scattered light!).  %Why does this mean that polarizable molecules can scatter light?  
%	Do you expect the intensity of the scattered light to increase or decrease as you increase the polarizability of the molecule?   Explain your group's reasoning in 2-3 complete sentences.  \label{\labelbase:ctq:polarizability}
%
%%%%%% Move this to exercises, and/or solution of previous Q (for instructor to discuss)
%%\begin{infobox}
%%
%%	If light with wavelength $\lambda$ and intensity $I_0$ is incident on a molecule with polarizability $\alpha$, the intensity of the scattered light $I_s$ is
%%	\begin{equation*}
%%		\frac{I_s}{I_0} = \frac{16 \alpha^2 \pi^4}{r^2 \lambda^4}
%%	\end{equation*}
%%	where $r$ is the distance from the molecule to the detector.
%%\end{infobox}
%		
%	\question In light scattering experiments, the polymers of interest are usually dissolved in a solvent.
%	
%		\begin{enumerate}	
%			\item \dots 
%		\end{enumerate}
%		
%	\question Explain, in 2-3 complete sentences, why we usually analyze the \emph{excess scattering}, $\dots$.  % could move to end and get them to the idea of dn/dc?
%	
%	\question In light scattering experiments, the wavelength of light is usually much longer than the size of the particles or the distances between them.
%	
%		GET AT IDEA THAT IT IS THE AVERAGE POLARIZABILITY THAT MATTERS - and then connect to refractive index?
%	
%\end{ctqs}
