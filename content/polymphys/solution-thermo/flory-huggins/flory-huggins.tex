%%%%%%%%%%%%%%%%%%%%%%%%%%%%%%%%%%%%%%%%%
%
% (c) 2019 by Jennifer Laaser
%
% This work is licensed under the Creative Commons Attribution-NonCommercial-ShareAlike 4.0 International License. To view a copy of this license, visit http://creativecommons.org/licenses/by-nc-sa/4.0/ or send a letter to Creative Commons, PO Box 1866, Mountain View, CA 94042, USA.
%
% The current source for these materials is accessible on Github: https://github.com/jlaaser/pogil-polymers
%
%%%%%%%%%%%%%%%%%%%%%%%%%%%%%%%%%%%%%%%%%

\renewcommand{\figpath}{content/polymphys/solutino-thermo/flory-huggins/figs}

\begin{activity}[Regular Solutions \& Flory-Huggins Theory]

\begin{instructornotes}

	This activity introduces students to key concepts related to polymer solutions, including ideal mixing, regular solution theory, and Flory-Huggins theory.
	
	After completing this activity, students will be able to:
			\begin{enumerate}
				\item ...
			\end{enumerate}
	This activity will prepare students for follow-up activities on phase diagrams for polymer solutions.
			
	\subsection*{Activity summary:}
	\begin{itemize}
		\item \textbf{Activity type:} Learning Cycle
		\item \textbf{Content goals:} Ideal mixing, regular solution theory, and Flory-Huggins theory
		\item \textbf{Process goals:} %https://pogil.org/uploads/attachments/cj54b5yts006cklx4hh758htf-process-skills-official-pogil-list-2015-original.pdf
			written communication, critical thinking, information processing
		\item \textbf{Duration:} TBD
		\item \textbf{Instructor preparation required:} none beyond knowledge of relevant content
		\item \textbf{Related textbook chapters:}
			\begin{itemize}
				\item \emph{Polymer Chemistry} (Hiemenz \& Lodge): sections 7.1-7.3
		\end{itemize}
	\end{itemize}

\end{instructornotes}

	%\textbf{Focus question:} Put a central question for the students to consider through this exercise here.

\begin{model}[Ideal Mixtures: Entropy of Mixing]

A simple model for mixing of two small-molecule liquids is shown below.  In this model, each molecule is shown as a circle, and we place them in a grid where each molecule takes up exactly one ``space'' in the grid.

Initially, the molecules of each type are isolated in their own containers:

After combining the two containers, the molecules may mix together:

The critical elements of this model are that
\begin{enumerate}[itemsep=0pt,topsep=-6pt]
	\item the number of molecules of each type does not change,
	\item the molecules each take up exactly the same volume (here, one square on the grid),
	\item the total volume after mixing is the sum of the two initial volumes, and
	\item the mixing is entirely random.
\end{enumerate}

\end{model}

\vspace{0.05in}
\begin{ctqs}

	\question Consider a simple, specific case of this model, as shown below:

		Here, $m_1=2$ (there are two molecules of type 1, shown as open circles) and $m_2=2$ (there are two molecules of type 2, shown as filled circles).
	
		\begin{enumerate}
			\item How many different ways can you distribute the two molecules of type 1 in their initial box, assuming the molecules are indistinguishable (you can't tell them apart)?  Sketch the possible configurations below.  Note that you may not need to use all of the boxes; cross off any you don't need.
		
			\item How many different ways can you distribute the two molecules of type 2 in their initial box, assuming the molecules are indistinguishable (you can't tell them apart)?  Sketch the possible configurations below. Again, you may not need to use all of the boxes.
			
			\item How many different ways can you distribute the four molecules in the final mixture, assuming that you can't distinguish type-1 molecules from other type-1 molecules, but you can distinguish type-1 molecules from type-2 molecules, and vice versa?  Again, you may not need to use all of the boxes.
		\end{enumerate}
		
	\question Recall that if $\Omega$ is the number of ways that the molecules in the system can be arranged, then the entropy of that system is given by
		\begin{equation*}
			S = k \ln \Omega
		\end{equation*} 
		where $k$ is the Boltzmann constant.  Using this information, and your answers from question 1, calculate
			
		\begin{enumerate}
			\item the entropy of the initial state ($S_{initial} = S_1 + S_2$):
			\item the entropy of the final (mixed) state, $S_{mixed}$):
			\item the change in entropy upon mixing ($\Delta S_{mix} = S_{mixed} - S_{initial}$): 
		\end{enumerate}
		
\end{ctqs}

\begin{infobox}
	Mathematically, the number of ways to fill $N$ boxes with $n$ indistinguishable molecules of type 1 and $N-n$ indistinguishable molecules of type 2 is
	\begin{equation*}
		\Omega = {N \choose n} = \frac{N!}{n!(N-n)!}
	\end{equation*}
	where $n! = n\cdot(n-1)\cdot(n-2)\cdot\dots\cdot 2 \cdot 1$.
	
	Note that by definition, $0!=1$.
\end{infobox}

\vspace{0.05in}
\begin{ctqs}
		
		\question For the initial state of the system shown in Model 1,
			\begin{enumerate}
				\item What are $N$ and $n$ for \emph{just} the type-1 molecules in the initial state?
				
					\begin{solution}[0.75in]
						\begin{align*}
							N = m_1 && \text{and} && n = m_1
						\end{align*}
					\end{solution}
					
				\item Using the mathematical expression given above, calculate $\Omega_1$, the number of configurations accessible to the type-1 molecules in the initial state.
				
					\begin{solution}[0.75in]
						\begin{equation*}
							\Omega_1 = {m_1 \choose m_1} = \frac{m_1!}{m_1! 0!} = 1
						\end{equation*}
					\end{solution}
				
				\item What do you expect $\Omega_2$, the number of configurations accessible to the type-2 molecules in the initial state, to be?
				
					\begin{solution}[0.75in]
						$\Omega_2$ should also equal 1.
					\end{solution}
					
				\item What is the total entropy of the initial state?
					
					\begin{solution}[1in]
						\begin{equation*}
						S_{initial} = k \ln \Omega_1 + k \ln \Omega 2 = k \ln 1 + k \ln 1 = 0
						\end{equation*}
					\end{solution}
			\end{enumerate}
		
		\question For the final (mixed) state of the system shown in Model 1, 
			\begin{enumerate}
				\item What are $N$ and $n$ for the final (mixed) state?
				
					\begin{solution}[0.75in]
						\begin{align*}
							N = m_1+m_2=m && \text{and} && n = m_1
						\end{align*}
					\end{solution}
					
				\item Write an expression for the number of configurations possible for the final (mixed) state in terms of $m_1$ and $m_2$.
				
					\begin{solution}[1in]
						\begin{equation*}
							\Omega_{mixed} = {(m_1+m_2) \choose m_1} = \frac{(m_1+m_2)!}{m_1! m_2!} = \frac{m!}{m_1! m_2!}
						\end{equation*}
					\end{solution}
					
				\item What is the entropy of the final (mixed) state?
				
					\begin{solution}[1in]
						\begin{align*}
							S_{mixed} &= k \ln \Omega_{mixed}\\
							&= \ln m! - \ln m_1! - \ln m_2!
						\end{align*}
					\end{solution}
				
				
			\end{enumerate}
		\question What is the entropy of mixing, $\Delta S_{mix} = S_{mixed} - S_{initial}$, for the system shown in Model 1?
				
					\begin{solution}[1in]
						\begin{align*}
							\Delta S_{mix} &= S_{mixed} - S_{initial}\\
							 &= \ln m! - \ln m_1! - \ln m_2! - 0 \\
							 &= \ln m! - \ln m_1! - \ln m_2!
						\end{align*}
					\end{solution}
\end{ctqs}

\begin{infobox}
	Logarithms of factorials can be approximated using Stirling's approximation,
	\begin{equation*}
		\ln N! \approx N \ln N - N \label{eqn:stirling}
	\end{equation*}
	Using this approximation, it is possible (after some algebra) to rewrite your expression for $\Delta S_{mix}$ as
	\begin{equation*}
		\Delta S_{mix} = -k\left(m_1 \ln\left(\frac{m_1}{m}\right) + m_2 \ln\left(\frac{m_2}{m}\right) \right)
	\end{equation*}
\end{infobox}

\begin{ctqs}
	\question As written, is $\Delta S_{mix}$ an \emph{extensive} property (which depends on the total number of molecules in the system) or an \emph{intensive} property (which does not depend on the total number of molecules present)?  Briefly explain your answer in 1-2 complete sentences.
	
	\question Usually, it is most convenient to divide by the total number of molecules, which leaves us with an intensive expression for the entropy,
		\begin{equation*}
			\Delta S_{mix}^{(int)} = \frac{1}{m} \Delta S_{mix}
		\end{equation*}
		Write an expression for $\Delta S_{mix}^{(int)}$ in terms of $m_1$, $m_2$, and $m$.
		
			\begin{solution}[1in]
			
				\begin{equation*}
					\Delta S_{mix}^{(int)} = -k\left(\frac{m_1}{m} \ln\left(\frac{m_1}{m}\right) + \frac{m_2}{m} \ln\left(\frac{m_2}{m}\right) \right)
				\end{equation*}
			\end{solution}
		
	\question We also often prefer to work in terms of \emph{mole fractions} rather than numbers of molecules. \label{ctq:Smixed}
	
		Rewrite your expression for $\Delta S_{mix}^{(int)}$ in terms of the mole fractions
		\begin{align*}
			x_1 = \frac{m_1}{m_1 + m_2} = \frac{m_1}{m} && \text{and} && x_2 = \frac{m_2}{m_1+m_2} = \frac{m_2}{m}
		\end{align*}
		
			\begin{solution}[1in]
			
				\begin{equation*}
					\Delta S_{mix}^{(int)} = -k\left(x_1 \ln x_1 + x_2 \ln x_2 \right)
				\end{equation*}
			\end{solution}
		
	\question The mole fractions, $x_1$ and $x_2$, must both be between 0 and 1.  In this case,
		\begin{enumerate}
			\item Will $\ln x_1$ (and $\ln x_2$) be positive or negative?
			\item Will $\Delta S_{mix}^{(int)}$ be positive or negative? \label{ctq:Spositive}
		\end{enumerate}
		
	\question Explain, in 1-2 complete sentences, why we say that mixing is an \emph{entropy-driven} process.
\end{ctqs}
	

\begin{model}[Real Mixtures: Enthalpy of Mixing]

In an \emph{ideal} mixture, the molecules do not interact with each other, and entropy is the only thermodynamic consideration that affects mixing.
In real mixtures, however, molecules do interact with each other, and we must take those interactions into account when determining whether mixing is favorable or unfavorable.

To incorporate the energetics of intermolecular interactions into our model, we must make two key assumptions:
\begin{enumerate}
	\item First, we assume that molecules interact only with their immediate neighbors, and that each interaction involves only two molecules.  The interaction energy between different types of pairs are as follows:
	
	IMAGE HERE
	
	\item Second, we assume that each molecule has some number of neighbors, $z$, which we refer to as the ``coordination number''.  For example,
		\begin{itemize}
			\item in a one-dimensional lattice, each molecule has exactly two neighbors, so $z=2$;
	
	IMAGE HERE
	
			\item in a two-dimensional lattice, each molecule has four immediate neighbors, so $z=4$;
	
	IMAGE HERE
	
			\item in a three-dimensional lattice, each molecule has six immediate neighbors, so $z=6$.
	
	IMAGE HERE
		\end{itemize}
\end{enumerate}

\end{model}

\begin{ctqs}

		\question Let's again start by consdering just the initial state of the system shown in Model 1, before any mixing takes place.
		
			\begin{enumerate}
				\item Consider the following type-1 molecule in its intial state, where it is surrounded only by other type-1 molecules:
		
		What is the total energy of the intermolecular interactions involving the central molecule?
		
					\begin{solution}[0.75in]
						Four surrounding molecules -> four interactions of strength $w_{11}$ -> total energy is $4 w_{11}$.
					\end{solution}
		
				\item More generally, if a molecule of type 1 is surrounded by $z$ other molecules of type 1, what is the total energy of the intermolecular interactions involving that molecule?
				
					\begin{solution}[0.75in]
						$z w_{11}$
					\end{solution}
		
				\item If the initial state contains $m_1$ such molecules, what is the total energy of the intermolecular interactions involving \emph{all} type-1 molecules in the initial state?
				
					\begin{solution}[0.75in]
						$m_1 z w_{11}$
					\end{solution}
		
				\item Similarly, if there are initially $m_2$ type-2 molecules, each surrounded by $z$ other type-2 molecules, what is the total energy of the intermolecular interactions involving \emph{all} type-2 molecules in the initial state?
				
					\begin{solution}[0.75in]
						$m_2 z w_{22}$
					\end{solution}
		
				\item What is the total energy (enthalpy) of the intermolecular interactions in the initial state, including contributions from both type-1 and type-2 molecules? \label{ctq:Hinitial}
				
					\begin{solution}[0.75in]
						$H_{initial} = m_1 z w_{11} + m_2 z w_{22}$
					\end{solution}
			\end{enumerate}
		
		\question Now, consider the mixed state, where a molecule of type-1 might be surrounded by some molecules of type 1 and some molecules of type 2, e.g.
		
			\begin{enumerate}
				\item If a molecule of type 1 is surrounded by $z$ other molecules, how many of those surrounding molecules are also type-1 molecules, assuming the mole fraction of molecules that are type 1 is $x_1$?
				
					\begin{solution}[0.75in]
						$z x_1$
					\end{solution}
				
				\item What is the total energy of interaction of the central molecule with its type-1 neighbors?
				
					\begin{solution}[0.75in]
						$z x_1 w_{11}$
					\end{solution}
				
				\item Similarly, how many of the molecule's neighbors are type-2 molecules, if the mole fraction of molecules that are type 2 is $x_2$?
				
					\begin{solution}[0.75in]
						$z x_2$
					\end{solution}
				
				\item What is the total energy of interaction of the central molecule with its type-2 neighbors?
				
					\begin{solution}[0.75in]
						$z x_2 w_{12}$
					\end{solution}
				
				\item What is the total energy of interaction of the central molecule with \emph{all} of its neighbors?
				
					\begin{solution}[0.75in]
						$z x_1 w_{11} + z x_2 w_{12}$
					\end{solution}
				
				\item If there are $m_1$ molecules of type 1 in the mixture, what is their total interaction energy with all of their neighbors?
				
					\begin{solution}[0.75in]
						$m_1(z x_1 w_{11} + z x_2 w_{12}) = m_1 z( x_1 w_{11} + x_2 w_{12})$
					\end{solution}
				
				\item By analogy, what is the total interaction energy of the $m_2$ type-2 molecules with all of their neighbors?
				
					\begin{solution}[0.75in]
					Swap the 1's and the 2's in the subscripts:
						$m_2(z x_2 w_{22} + z x_1 w_{12}) = m_2 z (x_2 w_{22} + x_1 w_{12})$
					\end{solution}
				
				\item Finally, what is the total energy (enthalpy) of the intermolecular interactions in the mixed state?\label{ctq:hmixed}
				
					\begin{solution}[1in]
						\begin{equation*}
							H_{mixed} = m_1 z( x_1 w_{11} + x_2 w_{12} + m_2 z (x_2 w_{22} + x_1 w_{12})
						\end{equation*}
					\end{solution}

			\end{enumerate}
			
		\question Combining the answers to questions \ref{ctq:Hinitial} and \ref{ctq:hmixed}, and doing a bit of algebra, we find that 
						\begin{align*}
							\Delta H_{mix} &= H_{initial} - H_{mixed} \\
								&= \frac{m_1 m_2}{m} z (2 w_{12} - w_{11} - w_{22})
						\end{align*}
		
				Note that this procedure, however, led us to double-count most of the interactions, so we need to divide by 2 to correct for this.  As with the entropy, it will also be useful to divide by $m$ to obtain an \emph{intensive} expression for the enthalpy of mixing.
		
			Make these two adjustments to write an expression for $\Delta H_{mix}^{(int)}$ in terms of $x_1$, $x_2$, $z$, $w_{11}$, $w_{22}$, and $w_{12}$.
			
			\begin{solution}[1.75in]
				\begin{align*}
					\Delta H_{mix}^{(int)} &= \frac{1}{2m}\frac{m_1 m_2}{m} z (2 w_{12} - w_{11} - w_{22})\\
						&= \frac{m_1}{m}\frac{m_2}{m} z \left( w_{12} - \frac{w_{11}}{2} - \frac{w_{22}}{2}\right)\\
						&= x_1 x_2 z \left( w_{12} - \frac{w_{11}}{2} - \frac{w_{22}}{2}\right)
				\end{align*}
			\end{solution}
			
\end{ctqs}
	
\begin{infobox}

The $w_{12}$, $w_{11}$, and $w_{22}$ terms can be combined into an \emph{exchange energy}, $\Delta w$, reflecting the aount of energy it takes to break up two sets of interactions between molecules of the same type and form new interactions between molecules of different types.

When $\Delta w$ is defined as
\begin{equation*}
	\Delta w = w_{12} - \frac{w_{11}}{2} - \frac{w_{22}}{2}
\end{equation*}
$\Delta H_{mix}^{(int)}$ can be written
\begin{equation*}
	\Delta H_{mix}^{(int)} = x_1 x_2 z \Delta w
\end{equation*}

Additionally, we typically normalize the total interaction energy between a particle and its neighbors ($z\Delta w$) by the thermal energy, $kT$.  This normalized quantity is defined as the \emph{interaction parameter}, $\chi$:
\begin{equation*}
	\chi = \frac{z\Delta w}{kT}
\end{equation*}
In terms of the interaction parameter, $\Delta H_{mix}^{(int)}$ can be written
\begin{equation*}
	\Delta H_{mix}^{(int)} = x_1 x_2 \chi kT
\end{equation*}

\end{infobox}
	
\begin{ctqs}
		\question Recalling that $\Delta G = \Delta H - T\Delta S$, combine this expression with your answer from question \ref{ctq:Smixed} to find an expression for $\Delta G_{mix}^{(int)}$.
			
			\begin{solution}[1in]
				\begin{align*}
					\Delta G_{mix}^{(int)} &= x_1 x_2 \chi kT - T(x_1 \ln x_1 + x_2 \ln x_2)
				\end{align*}
			\end{solution}
			
		\question In question \ref{ctq:Spositive}, we noted that $\Delta S_{mix}^{(int)}$ is always positive, so the entropic term \emph{always} favors mixing.
		
			Given that $\chi$ is \emph{usually} (although not always) positive, does the enthalpic term generally favor mixing, or oppose it?  Briefly explain your answer in 1-2 complete sentences.
			
			\begin{solution}[1in]
			\end{solution}
\end{ctqs}

\begin{model}[Polymer Solutions]

In Models 1 and 2, we learned that, for mixtures in which each molecule had exactly the same volume,
\begin{equation*}
	\frac{\Delta G_{mix}^{(int)}}{kT} = \underbrace{x_1 x_2 \chi}_{\text{enthalpic}} + \underbrace{x_1 \ln x_1 + x_2 \ln x_2}_{\text{entropic}}
\end{equation*}
where $x_1$ and $x_2$ are the \emph{mole fractions} of molecules of types 1 and 2, respectively.

For polymer solutions, however, the polymers have a much larger volume than the solvent.  Flory-Huggins theory allows us to correct for this difference by assuming that, on average, the conformations of the polymer chains do not change much between the bulk and the solution. In this case, we only have to worry about the entropy change from the center-of-mass placement of the polymer chains, not from the positions of the individual monomers.

In this case, for a solution of $m_1$ solvent molecules (each of which takes up 1 space), and $m_2$ polymer molecules (each of which take up $N$ spaces),  
\begin{equation*}
	\frac{\Delta G_{mix}^{(int)}}{kT} = \underbrace{\phi_1 \phi_2 \chi}_{\text{enthalpic}} + \underbrace{\phi_1 \ln \phi_1 + \frac{\phi_2}{N} \ln \phi_2}_{\text{entropic}}
\end{equation*}
where
\begin{align*}
	\phi_1 = \frac{m_1}{m_1 + N m_2} && \text{and} && \phi_2 = \frac{N m_2}{m_1 + N m_2}
\end{align*}
are the \emph{volume fractions} of sites occupied by solvent (1) and polymer (2).
\end{model}

\begin{ctqs}

		\question The contribution of the polymer molecules to the total entropy is $\frac{\phi_2}{N} \ln \phi_2$.  Is this term larger or smaller than the equivalent term in the small-molecule case when $x_1 = \phi_1$ and $x_2 = \phi_2$?
		
		\question Qualitatively, do you expect this $1/N$ term to make mixing more or less favorable than in the small-molecule case?
			
\end{ctqs}

\begin{exercises}

		\exercise Stirling's approximation (page \pageref{eqn:stirling}) is very useful in polymer physics and statistical mechanics.  Derive this formula by doing the following:
			\begin{enumerate}
				\item Rewrite $\ln N!$ as a summation of logarithms of individual numbers.  Remember that $\ln (a\cdot b) = \ln a + \ln b$, and write your answer in the form $\ln N!\sum_{i=1}^N \dots$.
				\item Use the trick $\sum_{i=1}^{N} f(i) \approx \int_1^N f(i)\,di$ to rewrite your expression as an integral.
				\item Evaluate the integral.
				\item Your answer won't agree exactly with the form of Stirling's approximation given in the exercise.  Why, in the limit that $N$ is very large, doesn't the discrepancy matter?
			\end{enumerate}
			
		\exercise In Model 3, we considered a polymer solution, in which a polymer with degree of polymerization $N$ is mixed with a small molecule solvent.
		
			What expression do you expect you would obtain for $\Delta G_{mix}^{(int)}/kT$ if we instead considered mixing of two polymers, one with degree of polymerization $N_1$ and the other with degree of polymerization $N_2$?
\end{exercises}
	
\end{activity}