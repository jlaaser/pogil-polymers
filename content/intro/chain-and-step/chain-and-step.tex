%%%%%%%%%%%%%%%%%%%%%%%%%%%%%%%%%%%%%%%%%
%
% (c) 2018 by Jennifer Laaser
%
% This work is licensed under the Creative Commons Attribution-NonCommercial-ShareAlike 4.0 International License. To view a copy of this license, visit http://creativecommons.org/licenses/by-nc-sa/4.0/ or send a letter to Creative Commons, PO Box 1866, Mountain View, CA 94042, USA.
%
% The current source for these materials is accessible on Github: https://github.com/jlaaser/pogil-polymers
%
%%%%%%%%%%%%%%%%%%%%%%%%%%%%%%%%%%%%%%%%%

\renewcommand{\figpath}{content/intro/chain-and-step/figs}
\renewcommand{\labelbase}{chain-and-step}

\begin{activity}{Introduction to Polymerization Mechanisms}
\begin{instructornotes}

	This activity introduces students to key concepts related to the chain-growth and step-growth polymerization mechanisms.
	
	After completing this activity, students will be able to:
			\begin{enumerate}
				\item \dots
			\end{enumerate}
			
	\subsection*{Activity summary:}
	\begin{itemize}
		\item \textbf{Activity type:} Learning Cycle
		\item \textbf{Content goals:} Basic polymerization mechanisms
		\item \textbf{Process goals:} %https://pogil.org/uploads/attachments/cj54b5yts006cklx4hh758htf-process-skills-official-pogil-list-2015-original.pdf
			written communication, critical thinking, information processing
		\item \textbf{Duration:} TBD
		\item \textbf{Instructor preparation required:} none beyond knowledge of relevant content
		\item \textbf{Related textbook chapters:}
			\begin{itemize}
				\item \emph{Polymer Chemistry} (Hiemenz \& Lodge): section TBD%1.7
			\end{itemize}
	\end{itemize}

\end{instructornotes}

\newcommand{\timeallowed}{3 minutes}

\begin{model}[Chain-Growth Polymerizations]
\label{\labelbase:mdl:chaingrowth}

	In one important type of polymerization, monomers cannot react with each other; they can only become part of a polymer chain if they react with the ``active'' end of an existing chain.
	
	Simulate this type of polymerization by doing the following:
	\begin{enumerate}
		\item Combine two bags of beads, of the same color, in your group's bin, and place it where everyone can reach it.
		\item Set a timer for \timeallowed.
		\item As soon as the timer is started, walk to the front of the room and get a special ``initiator'' bead from your instructor, then return to your desk. \emph{Note: each person in the group needs their own initiator bead!}
		\item Pick up beads one at a time from your group's bin, and add them to the chain according to the following rules:
			\begin{itemize}
				\item Attach the first bead directly to the initiator bead.
				\item All subsequent beads must be added to the growing end of the chain, \emph{not} to the initiator end.
			\end{itemize}
		\item Keep adding beads until all of the beads are gone or until the timer goes off.  If you finish before the timer goes off, record the time elapsed here: \rule{1in}{0.15mm}
	\end{enumerate}
	
	Use the chains generated in this simulation to answer this Model's CTQs, below.

\end{model}

\vspace{0.05in}
\begin{ctqs}

	\question Count the number of beads in each chain, and fill in the following table: \label{\labelbase:ctq:numbeadschain}
		
		\begin{center}
		\renewcommand{\arraystretch}{2}
			\begin{tabular}{|c|c|c|c|}
				\hline
				\textbf{Chain Length ($i$)} & \textbf{Number of Chains of Length $i$ ($n_i$)} & \hspace{0.75in} & \hspace{0.75in} \\\hline
				&&&\\\hline
				&&&\\\hline
				&&&\\\hline
				&&&\\\hline
				&&&\\\hline
				&&&\\\hline
			\end{tabular}
		\end{center}
		
		\emph{(Note: the two empty columns do not need to be filled in yet - you will get to them in CTQ \ref{\labelbase:ctq:Dchain}.)}
		
	\question Sketch a graph of $n_i$ vs $i$ and briefly describe its shape. \label{\labelbase:ctq:MWDchain}
	
		\begin{solution}[2.75in]
		\end{solution}
	
	\question Calculate $M_n$, $M_w$, and $D$ for this polymerization.  Assume that each bead represents a monomer with a molecular weight of 100~g/mol. \label{\labelbase:ctq:Dchain}
	
		\emph{Hint: you may find it useful use the extra columns in CTQ \ref{\labelbase:ctq:numbeadschain} to do some of the intermediate steps in the calculation, as we did in Activity 3.}
	
		\begin{solution}[2.75in]
		\end{solution}
	
	\question Based on your graph from CTQ \ref{\labelbase:ctq:MWDchain}, and the dispersity you calculated in CTQ \ref{\labelbase:ctq:Dchain}, would you characterize the molecular weight distribution generated by this polymerization as ``narrow'' or ``broad''?  %Briefly explain your reasoning.
	
		\begin{solution}[1in]
		\end{solution}
	
	\question If each member of your group had picked up their initiator bead at a different time, how would the molecular weight distribution and dispersity have changed?  Explain your group's reasoning in 2-3 complete sentences.
	
		\begin{solution}[2in]
		\end{solution}
	
	\question If everyone in your group had picked up their initiator bead at the same time, but each person added beads at a different speed, how would the molecular weight distribution and dispersity have changed?  Explain your group's reasoning in 2-3 complete sentences.
	
		\begin{solution}[2in]
		\end{solution}
		
\end{ctqs}

\begin{model}[Step-Growth Polymerizations]
\label{\labelbase:mdl:stepgrowth}

	Another type of polymerization uses monomers that each have two or more reactive sites.  Every molecule (monomer or polymer) may react with every other molecule (monomer or polymer) as long as they have complimentary reactive groups on their ends that are able to form a bond.
	
	Simulate this type of polymerization by doing the following:
	\begin{enumerate}
		\item Combine two bags of beads, of different colors, in your group's bin.  Remove any beads left over from the previous simulation if you have not done so already.
		\item Set a timer for \timeallowed, or however much time was required for the chain-growth polymerization in Model \ref{\labelbase:mdl:chaingrowth}, whichever is less.
		\item Start the timer!
		\item Start the polymerization by having each person pick up two beads \emph{of different colors}, connect them, and drop them back into the bin.
		\item Continue the polymerization by doing the following:
			\begin{itemize}
				\item Pick up two pieces (single beads, or strings of beads - try to choose as randomly as possible!) from the bin.
				\item If it is possible to connect the two pieces you picked up by connecting beads \emph{of different colors}, connect the two pieces and return them to the bin.
				\item Otherwise, return both pieces to the bin and try again.
			\end{itemize}
		\item Continue connecting beads until your timer goes off.
	\end{enumerate}
	
	Use the chains generated in this simulation to answer this Model's CTQs, below.

\end{model}
	
\begin{ctqs}

	\question Count the number of beads in each chain, and fill in the following table: \label{\labelbase:ctq:numbeadsstep}
		
		\begin{center}
		\renewcommand{\arraystretch}{2}
			\begin{tabular}{|c|c|c|c|}
				\hline
				\textbf{Chain Length ($i$)} & \textbf{Number of Chains of Length $i$  ($n_i$)} & \hspace{0.75in} & \hspace{0.75in} \\\hline
				&&&\\\hline
				&&&\\\hline
				&&&\\\hline
				&&&\\\hline
				&&&\\\hline
				&&&\\\hline
				&&&\\\hline
				&&&\\\hline
				&&&\\\hline
				&&&\\\hline
				&&&\\\hline
				&&&\\\hline
				&&&\\\hline
				&&&\\\hline
				&&&\\\hline
				&&&\\\hline
			\end{tabular}
		\end{center}
		
	\question Sketch a graph of $n_i$ vs $i$ and briefly describe its shape. \label{\labelbase:ctq:MWDstep}
	
		\begin{solution}[3in]
		\end{solution}
	
	\question Calculate $M_n$, $M_w$, and $D$ for this polymerization.  Assume that each bead represents a monomer with a molecular weight of 100~g/mol. \label{\labelbase:ctq:Dstep}
	
		\begin{solution}[3in]
		\end{solution}
	
	\question Based on your graph from CTQ \ref{\labelbase:ctq:MWDstep}, and the dispersity you calculated in CTQ \ref{\labelbase:ctq:Dstep}, would you characterize the molecular weight distribution generated by this polymerization as ``narrow'' or ``broad''?  %Briefly explain your reasoning.
	
		\begin{solution}[1in]
		\end{solution}
	
	\question Suppose you had kept going with the simulation until no more  connections could be made.  What would be the result?  Explain your group's reasoning in 2-3 complete sentences.
	
		\begin{solution}[2in]
		\end{solution}
	
	\question If there had instead been significantly more beads of one color than the other, how would the final distributions of chains be different?  Explain your group's reasoning in 2-3 complete sentences.
	
		\begin{solution}[2in]
		\end{solution}
	
\end{ctqs}

\begin{infobox}
	The type of polymerization you simulated in \ref{\labelbase:mdl:chaingrowth} is called a \emph{chain-growth} polymerization, while the type of polymerization you simulated in  \ref{\labelbase:mdl:stepgrowth} is called a \emph{step-growth} polymerization.
	
	All of the polymerization methods we discuss in class will fall into one of these two categories, and it is useful to recognize the advantages and disadvantages of each approach.
\end{infobox}

\begin{ctqs}
	\question Which polymerization mechanism (chain-growth or step-growth) gave longer polymers in the time allowed for the simulation?
	
		\begin{solution}[0.75in]
		\end{solution}
	
	\question Which method would give the longest polymers if there were an infinite amount of time?  Explain your group's reasoning in 1-2 complete sentences.
	
		\begin{solution}[1.75in]
		\end{solution}
	
	\question Which polymerization mechanism gave a larger dispersity?  In 2-3 complete sentences, explain why your group thinks this happened.
	
		\begin{solution}[1.75in]
		\end{solution}
	
	\question If you needed to make a batch of 700~kg/mol polyethylene, which type of polymerization would you choose?  Explain your group's choice in 1-2 complete sentences.
	
		\begin{solution}[1.75in]
		\end{solution}
	
\end{ctqs}

\begin{exercises}

		\exercise In a brief paragraph, describe the key similarities and differences between the chain-growth and step-growth polymerization mechanisms.
		
		\exercise Summarize the key conditions necessary for a chain-growth polymerization to achieve a narrow molecular weight distribution.
			
\end{exercises}
	
\end{activity}