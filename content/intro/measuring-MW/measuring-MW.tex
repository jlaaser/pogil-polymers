%%%%%%%%%%%%%%%%%%%%%%%%%%%%%%%%%%%%%%%%%
%
% (c) 2022 by Jennifer Laaser
%
% This work is licensed under the Creative Commons Attribution-NonCommercial-ShareAlike 4.0 International License. To view a copy of this license, visit http://creativecommons.org/licenses/by-nc-sa/4.0/ or send a letter to Creative Commons, PO Box 1866, Mountain View, CA 94042, USA.
%
% The current source for these materials is accessible on Github: https://github.com/jlaaser/pogil-polymers
%
%%%%%%%%%%%%%%%%%%%%%%%%%%%%%%%%%%%%%%%%%

\renewcommand{\figpath}{content/intro/measuring-MW/figs}
\renewcommand{\labelbase}{measuring-MW}

\begin{activity}{Measuring Molecular Weight}
\label{\labelbase}

\begin{instructornotes}

	This activity introduces students to key concepts related to methods used to measure the molecular weights and molecular weight distributions of polymers.
	
	After completing this activity, students will be able to:
			\begin{enumerate}
				\item Use end-group analysis via NMR and UV-Vis to calculate the number-average molecular weight of a polymer
				\item Qualitatively interpret the relative molecular weights and dispersities of polymers measured by SEC
				\item Interpret MALDI spectra in terms of the repeat unit mass and frequency of chain lengths, and calculate molecular weights and dispersities of polymer samples from MALDI data
			\end{enumerate}
			
	\subsection*{Activity summary:}
	\begin{itemize}
		\item \textbf{Activity type:} Learning Cycle
		\item \textbf{Content goals:} Measuring molecular weights
		\item \textbf{Process goals:} %https://pogil.org/uploads/attachments/cj54b5yts006cklx4hh758htf-process-skills-official-pogil-list-2015-original.pdf
			\begin{itemize}
				\item Interpreting graphs and numeric data
				\item Written and oral communication of reasoning
			\end{itemize}
		\item \textbf{Duration:} TBD
		\item \textbf{Instructor preparation required:} none beyond knowledge of relevant content
		\item \textbf{Related textbook chapters:}
			\begin{itemize}
				\item \emph{Polymer Chemistry} (Hiemenz \& Lodge): section 1.8
			\end{itemize}
		%\item \textbf{Facilitation notes:}
		%	\begin{itemize}
		%		\item \dots
		%	\end{itemize}
	\end{itemize}
	
\end{instructornotes}




\begin{model}[End Group Analysis]
\label{\labelbase:mdl:endgrpanalysis}

	\dots

\end{model}


\begin{ctqs}

	\question \dots
	
\end{ctqs}

\begin{infobox}

	\dots

\end{infobox}

\begin{ctqs}
		
	\question \dots
	
\end{ctqs}



\begin{model}[Size-Exclusion Chromatography]
\label{\labelbase:mdl:SEC}
	
	\dots

\end{model}

\begin{ctqs}

	\question \dots
	
\end{ctqs}



\begin{model}[MALDI Mass Spectrometry]
	\label{\labelbase:mdl:MALDI}

	\dots

\end{model}

\begin{ctqs}

	\question \dots
		
\end{ctqs}


\begin{exercises}

	\exercise \dots
	
\end{exercises}


	
\end{activity}