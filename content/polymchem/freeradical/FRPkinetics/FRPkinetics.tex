%%%%%%%%%%%%%%%%%%%%%%%%%%%%%%%%%%%%%%%%%
%
% (c) 2019 by Jennifer Laaser
%
% This work is licensed under the Creative Commons Attribution-NonCommercial-ShareAlike 4.0 International License. To view a copy of this license, visit http://creativecommons.org/licenses/by-nc-sa/4.0/ or send a letter to Creative Commons, PO Box 1866, Mountain View, CA 94042, USA.
%
% The current source for these materials is accessible on Github: https://github.com/jlaaser/pogil-polymers
%
%%%%%%%%%%%%%%%%%%%%%%%%%%%%%%%%%%%%%%%%%

\renewcommand{\figpath}{content}
\renewcommand{\labelbase}{FRPkinetics}

\begin{activity}[Kinetics of Free-Radical Polymerization]

\begin{instructornotes}
	This activity introduces students to concepts related to the kinetics of free-radical polymerization.
	
	After completing this activity, students will be able to:
	\begin{enumerate}
		\item Write kinetic equations for each of the major steps of a free-radical polymerization
		\item Calculate the kinetic chain length for a free-radical polymerization
	\end{enumerate}
	
	\subsection*{Activity summary:}
	\begin{itemize}
		\item \textbf{Activity type:} Learning Cycle
		\item \textbf{Content goals:} Kinetics of free-radical polymerization
		\item \textbf{Process goals:} %https://pogil.org/uploads/attachments/cj54b5yts006cklx4hh758htf-process-skills-official-pogil-list-2015-original.pdf
			written communication, critical thinking, information processing
		\item \textbf{Duration:} TBD
		\item \textbf{Instructor preparation required:} none beyond knowledge of relevant content
		\item \textbf{Related textbook chapters:}
			\begin{itemize}
				\item \emph{Polymer Chemistry} (Hiemenz \& Lodge): section NNN
			\end{itemize}
		%\item \textbf{Facilitation notes:}
		%	\begin{itemize}
		%		\item \dots
		%	\end{itemize}
	\end{itemize}
	
\end{instructornotes}


\begin{model}[Kinetic Equations]
	\label{\labelbase:mdl:kineticeqns}

	SUMMARIZE KINETICS EQUATIONS FOR FIRST ORDER, SECOND ORDER, ETC.

\end{model}


\begin{ctqs}

	\question The first step in a free-radical polymerization is \emph{initiation}:
	
		Using the information in Model \ref{\labelbase:mdl:kineticeqns}, write an expression for the rate at which new initiator radicals are generated, $\frac{d[I^{\bullet}]}{dt}$, in terms of the concentration of initiator ($[Init]$) and dissociation constant $k_d$.
		
	\question Some of the initiator radicals then attack monomers to form growing polymer chains:
	
		If a fraction $f$ of the initiator radicals successfully generate new polymer chains, what is the overall initiation rate, $\frac{d[P^{\bullet}]}{dt}$?
		
	\question The second step in a free-radical polymerization is propagation:
	
		Write an expression for $-\frac{d[M]}{dt}$, the rate at which monomers are used up.
		
	\question The final step in a free-radical polymerization is termination:
	
		Write an expression for $-\frac{d[P^{\bullet}]}{dt}$, the rate at which polymer radicals are removed from the reaction.

\end{ctqs}



\begin{model}[Summary of Free-Radical Kinetics]
\label{\labelbase:mdl:kineticssummary}

	The major reaction steps in free-radical polymerization and their rate laws are summarized below:
	
	IMAGE

\end{model}

\begin{ctqs}

	\question To simplify the kinetic equations shown in Model \ref{\labelbase:mdl:kineticeqns}, we assume that the concentration of radicals reaches a ``steady state'', where $[P^{\bullet}]$ is constant.
	
		\begin{enumerate}
			\item One way to ensure that $[P^{\bullet}]$ reaches a steady state is to require that the rate at which radicals are generated is equal to the rate at which they are consumed.
			
				Which two rates ($R_i$, $R_p$, and/or $R_t$) must be equal in order for this to be true?
				
			\item Set the two rates identified in the previous question equal to each other and solve for $[P^{\bullet}]$ in terms of $f$, $k_d$, $k_t$, and $[Init]$.
			
			\item Finally, plug your answer into the equation for $R_p$ to find an expression for the polymerization rate, $R_p$, in terms of $f$, $k_d$, $k_p$, $k_t$, $[M]$, and $[Init]$.
		\end{enumerate}
	
	\question WALK THROUGH KINETIC CHAIN LENGTH
	
\end{ctqs}

		
\begin{model}
	Something about chain transfer?
\end{model}



\begin{exercises}

	\exercise In Model \ref{\labelbase:mdl:kineticssummary}, you showed that
		\begin{align*}
			R_p = -\frac{d[M]}{dt} = k_p [M] \left(\frac{f k_d}{k_t}\right)^{1/2}[Init]^{1/2}
		\end{align*}
		Show that if the initial concentration of initiator is $[Init]_0$ and the initial concentration of monomer is $[M]_0$, then the time-dependent monomer concentration obeys
		\begin{align*}
			-\ln\frac{[M]}{[M]_0} = \left(\frac{k_p}{k_t^{1/2}}\right)\left(\frac{f[I]_0}{k_d}\right)^{1/2}\left(1 - e^{-k_d t/2}\right)
		\end{align*}
	
\end{exercises}


\begin{problems}

	\problem First exercise
	\problem Second exercise
	
\end{problems}


	
\end{activity}