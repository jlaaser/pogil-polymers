%%%%%%%%%%%%%%%%%%%%%%%%%%%%%%%%%%%%%%%%%
%
% (c) 2019 by Jennifer Laaser
%
% This work is licensed under the Creative Commons Attribution-NonCommercial-ShareAlike 4.0 International License. To view a copy of this license, visit http://creativecommons.org/licenses/by-nc-sa/4.0/ or send a letter to Creative Commons, PO Box 1866, Mountain View, CA 94042, USA.
%
% The current source for these materials is accessible on Github: https://github.com/jlaaser/pogil-polymers
%
%%%%%%%%%%%%%%%%%%%%%%%%%%%%%%%%%%%%%%%%%

\renewcommand{\figpath}{content/polymchem/freeradical/FRPthermo/figs}
\renewcommand{\labelbase}{FRPthermo}

\begin{activity}[Thermodynamics of Free-Radical Polymerization]

\begin{instructornotes}
	This activity introduces students to concepts related to the thermodynamics of free-radical polymerization.
	
	After completing this activity, students will be able to:
	\begin{enumerate}
		\item ...
	\end{enumerate}
	
	\subsection*{Activity summary:}
	\begin{itemize}
		\item \textbf{Activity type:} Learning Cycle
		\item \textbf{Content goals:} Thermodynamics of free-radical polymerization
		\item \textbf{Process goals:} %https://pogil.org/uploads/attachments/cj54b5yts006cklx4hh758htf-process-skills-official-pogil-list-2015-original.pdf
			written communication, critical thinking, information processing
		\item \textbf{Duration:} TBD
		\item \textbf{Instructor preparation required:} none beyond knowledge of relevant content
		\item \textbf{Related textbook chapters:}
			\begin{itemize}
				\item \emph{Polymer Chemistry} (Hiemenz \& Lodge): section NNN
			\end{itemize}
		%\item \textbf{Facilitation notes:}
		%	\begin{itemize}
		%		\item \dots
		%	\end{itemize}
	\end{itemize}
	
\end{instructornotes}


\begin{model}[Thermodynamics of Propagation]
	\label{\labelbase:mdl:propthermo}

	(include a rxn scheme for propagation)
	
\end{model}


\begin{ctqs}

	\question \dots
	
	\question Write an appropriate expression for $\Delta G$ in terms of $\Delta H$ and $\Delta S$.  What sign does $\Delta G$ need to have in order for the reaction to be favorable? \label{\labelbase:ctq:DeltaG}

\end{ctqs}




\begin{model}[Equilibrium of Propagation]
	\label{\labelbase:mdl:propequilib}

	\dots
	
\end{model}


\begin{ctqs}

	\question Write an appropriate equilibrium constant for this reaction, in terms of [\ce{P_n^.}], [\ce{P_{n+1}^.}], and [M].
	
	\question In most polymerizations, [\ce{P_n^.}]$\approx$[\ce{P_{n+1}^.}].  Use this approximation to rewrite your equilibrium constant from the previous problem as a function of [M] only.
	
	%\question I'd really like to get at the idea that if you start the rxn above the equlibrium monomer concentration, then it will run until it reaches equilibrium, while if you start the rxn below the equilibrium monomer concentration, the reaction simply won't go.  But I am not really sure how to work this in, or where in these CTQs it belongs.  Maybe:
	
	%\question Suppose you have a polymerization reaction whose equilibrium monomer concentration is $[M]_{eq}$.  
	%	\begin{enumerate}
	%		\item If you start a reaction with a concentration of monomer $[M] > M_{eq}$, will any polymer form?  Why or why not?
	%		\item If you start a reaction with a concentration of monomer $[M] < M_{eq}$, will any polymer form?  Why or why not?
	%		\item what happens to [M] over the course of the reaction?
	%	\end{enumerate}
	% not sure I like that, but may be a start.  Or, could leave the CTQs as is, and put this into the exercises for this activity?  E.g. write it as, "In Model 2, we looked at ceiling temp from the perspective of maximum temp at which rxn will run.  However, we could also look at it from the perspective of concentration of monomer at which rxn stops." etc.
	
	\question Recall that a reaction is in equilibrium when $K = e^{-\Delta G^\circ/RT}$.  Use this relationship, and your answer to CTQ \ref{\labelbase:ctq:DeltaG}, write an appropriate expression relating the equilibrium monomer concentration to the $\Delta H$ and $\Delta S$ of the propagation reaction.
	
	\question Solve for the temperature, $T_c$, at which the reaction is in equilibrium with a monomer concentration [M].
	
	\question Recall that the reaction in Model \ref{\labelbase:mdl:propequilib} is exothermic.  If you increase the temperature beyond $T_c$, which side of the reaction (reactants or products) will be favored?

	\question Explain why $T_c$ (which we refer to as the ``ceiling temperature'' of the reaction) can be considered to be the maximum temperature at which polymer will form.

\end{ctqs}



\begin{model}[Changes to Reaction Rates During Polymerization]
\label{\labelbase:mdl:rxnrates}

	In our derivation of the kinetics of free radical polymerization, we assumed that the rate constants ($k_d$, $k_p$, and $k_t$) were constant.  In real polymerizations, however, this does not always hold true.
	
	\dots
	

\end{model}

\begin{ctqs}

	\question 
		
\end{ctqs}


\begin{exercises}

	\exercise \dots
	
\end{exercises}


%\begin{problems}
%
%	\problem First exercise
%	\problem Second exercise
%	
%\end{problems}


	
\end{activity}