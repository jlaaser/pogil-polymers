%%%%%%%%%%%%%%%%%%%%%%%%%%%%%%%%%%%%%%%%%
%
% (c) 2022 by Jennifer Laaser
%
% This work is licensed under the Creative Commons Attribution-NonCommercial-ShareAlike 4.0 International License. To view a copy of this license, visit http://creativecommons.org/licenses/by-nc-sa/4.0/ or send a letter to Creative Commons, PO Box 1866, Mountain View, CA 94042, USA.
%
% The current source for these materials is accessible on Github: https://github.com/jlaaser/pogil-polymers
%
%%%%%%%%%%%%%%%%%%%%%%%%%%%%%%%%%%%%%%%%%

\renewcommand{\figpath}{content/polymchem/livingpolyms/ionic-polyms/figs}
\renewcommand{\labelbase}{ionic-polyms}

\begin{activity}{Ionic Polymerizations}

\begin{instructornotes}
	This activity introduces students to concepts related to anionic and cationic polymerizations.
	
	After completing this activity, students will be able to:
	\begin{enumerate}
		\item \dots
	\end{enumerate}
	
	\subsection*{Activity summary:}
	\begin{itemize}
		\item \textbf{Activity type:} Learning Cycle
		\item \textbf{Content goals:} Ionic Polymerizations
		\item \textbf{Process goals:} %https://pogil.org/uploads/attachments/cj54b5yts006cklx4hh758htf-process-skills-official-pogil-list-2015-original.pdf
			\begin{itemize}
				\item Reading and interpreting reaction mechanisms
				\item \dots
				\item Oral and written communication of reasoning
			\end{itemize}
		\item \textbf{Duration:} 45 minutes, including class discussion
		\item \textbf{Instructor preparation required:} none beyond knowledge of relevant content
		\item \textbf{Related textbook chapters:}
			\begin{itemize}
				\item \emph{Polymer Chemistry} (Hiemenz \& Lodge): section 4.3
			\end{itemize}
		%\item \textbf{Facilitation notes:}
		%	\begin{itemize}
		%		\item \dots
		%	\end{itemize}
	\end{itemize}
	
\end{instructornotes}


\begin{model}[Anionic Polymerization]
	\label{\labelbase:mdl:anionic}


	
\end{model}


\begin{ctqs}

	\question ...

\end{ctqs}

\begin{infobox}

	\dots
	
\end{infobox}

\begin{ctqs}
	
	\question \dots
		
\end{ctqs}


\begin{exercises}

	\exercise \dots
	
\end{exercises}


%\begin{problems}
%
%	\problem First exercise
%	\problem Second exercise
%	
%\end{problems}


	
\end{activity}