%%%%%%%%%%%%%%%%%%%%%%%%%%%%%%%%%%%%%%%%%
%
% (c) 2018 by Jennifer Laaser
%
% This work is licensed under the Creative Commons Attribution-NonCommercial-ShareAlike 4.0 International License. To view a copy of this license, visit http://creativecommons.org/licenses/by-nc-sa/4.0/ or send a letter to Creative Commons, PO Box 1866, Mountain View, CA 94042, USA.
%
% The current source for these materials is accessible on Github: https://github.com/jlaaser/pogil-polymers
%
%%%%%%%%%%%%%%%%%%%%%%%%%%%%%%%%%%%%%%%%%

\renewcommand{\figpath}{content/polymchem/stepgrowth/dispersity/figs}
\renewcommand{\labelbase}{step-dispersity}

\begin{activity}{Molecular Weight Distributions in Step-Growth Polymerizations}

\begin{instructornotes}

	This activity introduces students to key concepts related to the molecular weight distributions obtained in step-growth polymerizations.
	
	After completing this activity, students will be able to:
			\begin{enumerate}
				\item Calculate the fraction of polymer chains with length $i$ in a step-growth polymerization,
				\item Describe, qualitatively, the chain length distribution and how it changes with extent of reaction,
				\item Calculate the expected dispersity for step-growth polymerizations, and
				\item Explain why the limiting dispersity for a step-growth polymerization is 2.
			\end{enumerate}
			
	\subsection*{Activity summary:}
	\begin{itemize}
		\item \textbf{Activity type:} Learning Cycle
		\item \textbf{Content goals:} Molecular weight distributions and dispersity in step-growth polymerizations
		\item \textbf{Process goals:} %https://pogil.org/uploads/attachments/cj54b5yts006cklx4hh758htf-process-skills-official-pogil-list-2015-original.pdf
			written communication, critical thinking, information processing
		\item \textbf{Duration:} 25-30 minutes, including time for discussion
		\item \textbf{Instructor preparation required:} none beyond knowledge of relevant content
		\item \textbf{Related textbook chapters:}
			\begin{itemize}
				\item \emph{Polymer Chemistry} (Hiemenz \& Lodge): section 2.4
			\end{itemize}
		\item \textbf{Facilitation notes:}
			\begin{itemize}
				\item The process described in Model 1 depicts step-growth polymerizations as a monomer-by-monomer addition process.  Although this simplification was necessary to make the model accessible to students, it is not strictly accurate, since oligomers can react with each other in step-growth polymerizations (and not just with monomers).
			\end{itemize}
	\end{itemize}

\end{instructornotes}

	%\textbf{Focus question:} Put a central question for the students to consider through this exercise here.

\begin{model}[Probabilities of Forming Different Chain Lengths]
\label{\labelbase:mdl:probabilities}

Recall that in a step-growth polymerization of AB-type monomers, the extent of reaction $p$ gives the fraction of `A' groups that have reacted.

In terms of probabilities, the \emph{probability} that a particular `A' group reacts (turns into an `a' group) is $p$, while the probability that it \emph{does not} react (remains an `A' group) is $1-p$:

	\centerline{\includegraphics[width=0.7\textwidth]{\figpath/model1-rxn}}

	Each molecule has some combination of unreacted `A' groups and reacted `a' groups, each of which occurs with probability $p$ or $(1-p)$ as appropriate:

	\centerline{\includegraphics[width=0.5\textwidth]{\figpath/model1-molecule}}
	
	The \emph{total} probability of forming this molecule is just the product of the probabilities for each group:
	\begin{equation*}
		P(molecule) = \underbrace{(p \times p \times p \times \dots)}_{\text{one factor of p for each reacted `a' group}} \times \underbrace{((1-p) \times (1-p) \times \dots)}_{\text{one factor of (1-p) for each unreacted `A' group}} 
	\end{equation*}
or, more concisely,
	\begin{equation*}
		P(molecule) = p^\text{(number of reacted `a' groups)}\times(1-p)^\text{(number of unreacted `A' groups)}
	\end{equation*}


\end{model}

\vspace{0.05in}
\begin{ctqs}
	
	\question Consider an AbabaB trimer:
		\begin{enumerate}
			\item How many monomers came together to form this molecule (that is, what is its degree of polymerization)?
			
				\begin{solution}[0.5in]
					3
				\end{solution}
				
			\item How many \emph{unreacted} `A' groups are in this molecule?
			
				\begin{solution}[0.5in]
					1
				\end{solution}
				
			\item How many \emph{reacted} `a' groups are in this molecule?
			
				\begin{solution}[0.5in]
					2
				\end{solution}
				
			\item What is the probability of forming an AbabaB trimer?
			
				\begin{solution}[0.5in]
					$P(AbabaB) = p^\text{number of reacted a groups}(1-p)^\text{number of unreacted A groups} = p^2(1-p)^1 = p^2(1-p)$
				\end{solution}
				
		\end{enumerate}
		
	\question More generally, consider a molecule with degree of polymerization $i$ (that is, a molecule that was made by linking together $i$ AB-type monomers).
	
		\begin{enumerate}
		
			\item How many \emph{unreacted} `A' groups are in this molecule?
			
				\begin{solution}[0.5in]
					1
				\end{solution}
				
			\item How many \emph{reacted} `a' groups are in this molecule?
			
				\begin{solution}[0.5in]
					i-1
				\end{solution}
				
			\item What is the probability of forming a molecule with degree of polymerization $i$?
			
				\begin{solution}[1in]
					$P(i-mer) = p^\text{number of reacted a groups}(1-p)^\text{number of unreacted A groups} = p^{i-1}(1-p)^1 = p^{i-1}(1-p)$
				\end{solution}
			
		\end{enumerate}
		
\end{ctqs}

\begin{infobox}
	The \emph{probability} that a molecule is has degree of polymerization $i$ is the same as the \emph{fraction of molecules} that have degree of polymerization $i$.
\end{infobox}


\begin{ctqs}
	
	\question Complete the following statement:
	
		``The fraction of molecules, $x_i$, that have degree of polymerization $i$ is \line(1,0){50}.''
	
		\begin{solution}[0.5in]
		
			$x_i = p^{i-1}(1-p)$
			
		\end{solution}
		
\end{ctqs}

\begin{model}[Chain Length Distributions]
\label{\labelbase:mdl:dist}

	The following table gives selected values of $x_i$ calculated at two different extents of reaction, $p$, using the expression you derived in Model \ref{\labelbase:mdl:probabilities}:
	
		\begin{center}
			\renewcommand{\arraystretch}{1.5}
			\begin{tabular}{ccc}
				\hline
				\textbf{~~$i$~~} & ~~~$x_i$ when $p=0.5$~~~ & ~~~$x_i$ when $p=0.9$~~~ \\\hline
				1 & 0.5 & 0.1 \\
				2 & 0.25 & 0.09 \\
				3 & 0.125 & 0.08 \\
				5 & 0.0313 & 0.065 \\
				10 & 9.7x10$^{-4}$ & 0.0387 \\
				15 & 3.1x10$^{-5}$ & 0.0229 \\
				20 & 9.5x10$^{-7}$ & 0.0135 \\\hline
			\end{tabular}
		\end{center}
		
		\vspace{0.1in}

\end{model}
	
\begin{ctqs}

	\question Plot the data given in Model \ref{\labelbase:mdl:dist} on the following axes.  Make sure to use a different symbol for points corresponding to $p=0.5$ than for the points corresponding to $p=0.9$.
	
		\begin{solution}[3.25in]
			\studentdisplay{
				\centerline{\includegraphics[width=0.8\textwidth]{\figpath/model2-xi-axes.pdf}}
			}
			\instructordisplay{
				\centerline{\includegraphics[width=0.8\textwidth]{\figpath/model2-xi-plotted.pdf}}
			}
		\end{solution}
	
	\clearpage
	\question How are the plots for $p=0.5$ and $p=0.9$ similar, and how are they different?  Briefly describe your observations in 2-3 complete sentences.
	
		\begin{solution}[1.5in]
		
			Both of these plots decrease exponentially toward zero with increasing values of $i$.  However, the plot for $p=0.5$ decreases much faster, and a higher fraction of the molecules have very short chain lengths, than in the case where $p=0.9$.
		\end{solution}
	
	\question What is the \emph{most probable} chain length for each value of $p$?  Briefly explain your answer in 1-2 complete sentences.
	
		\begin{solution}[1.5in]
		
			The most probable chain length is just the one with the highest value of $x_i$.  Thus, the most probable chain length is $i=1$ for both values of $p$.
		
		\end{solution}
	
	\question Can the fraction of chains with length $i+1$ ever be \emph{greater} than the fraction of chains with length $i$?  Justify your answer in 2-3 complete sentences.
	
		\begin{solution}[1.5in]
		
			No, the fraction of chains with length $i+1$ can never be greater than the mole fraction of chains with length $i$.  This is because for each additional monomer, we pick up another factor of $p$; since $p$ is always less than one, $x_{i+1}$ will always be less than $x_i$.
			
			In mathematical terms, $x_i$ decreases monotonically with increasing chain length $i$.
		
		\end{solution}
	
\end{ctqs}

\clearpage
\begin{model}[$M_n$ and $M_w$ for Step-Growth Polymerizations]
\label{\labelbase:mdl:MwMn}

	To calculate $M_n$ and $M_w$, we need to know $n_i$, or the total number of chains with $i$ monomers.
	
	If we started with $v_A^0$ monomers, then when the extent of reaction is equal to $p$, there will be $(1-p)v_A^0$ unreacted A groups left.  Recalling that the number of unreacted A groups is equal to the number of molecules in the reaction mixture, this lets us write
	\begin{align*}
		n_i = \text{(fraction of molecules that }&\text{have length }i\text{) x (number of molecules in reaction mixture)}\\
			%&= (x_i)((1-p)v_A^0)\\
			&= \left(p^{i-1}(1-p)\right)\left((1-p)v_A^0\right)\\
			&= p^{i-1}(1-p)^2v_A^0
	\end{align*}
	
	If we plug this expression into our equation for $M_n$, we get
	\begin{equation*}
		M_n = \frac{\sum_i n_i M_i}{\sum_i n_i} %= \frac{\sum_i p^{i-1}(1-p)^2 v_A^0 i M_0}{\sum_i p^{i-1}(1-p)^2 v_A^0} 
		= M_0\frac{\sum_i p^{i-1}(1-p)^2 i }{\sum_i p^{i-1}(1-p)^2}
	\end{equation*}
	where $M_0$ is the molecular weight of the monomer ($M_i = M_0 i$).
	
	\vspace{0.25in}
	
	Evaluating these sums (see Exercise \ref{labelbase:exc:Mn}), we obtain
	\begin{align*}
		M_n = \frac{M_0}{1-p} && \text{or} && N_n = \frac{M_n}{M_0} = \frac{1}{1-p}
	\end{align*}
	which is exactly what we came up with in our previous activity on degree of polymerization.
	
	\vspace{0.25in}
	Similarly, plugging the above expression for $n_i$ into our expression for $M_w$ and evaluating the sums, we obtain
	\begin{align*}
		M_w = \frac{\sum_i n_i M_i^2}{\sum_i n_i M_i} = M_0\frac{1+p}{1-p} && \text{or} && N_w = \frac{M_w}{M_0} = \frac{1+p}{1-p}
	\end{align*}

\end{model}

\begin{ctqs}
		\question Calculate the dispersity for a step-growth reaction with extent of reaction $p$.
		
			\begin{solution}[1.95in]
			
				\begin{equation*}
					\text{\DJ} = \frac{M_w}{M_n} = \frac{M_0\frac{1+p}{1-p}}{M_0\frac{1}{1-p}} = 1+p
				\end{equation*}
			\end{solution}
			
			
		\question What is the value of the dispersity when $p=0$?  Briefly comment on whether or not this answer makes sense.
		
			\begin{solution}[1.5in]
			
				When $p=0$, $\text{\DJ}=1+0 = 1$.  This does make sense: when the extent of reaction is zero, no reactions have taken place, and the reaction mixture contains only monomers.  Since all of the molecules in the mixture are thus identical (and exactly the same size), the dispersity is 1 - the mixture is perfectly monodisperse.
			
			\end{solution}
			
			
		\question What is the value of the dispersity when $p=1$?
		
			\begin{solution}[1in]
			
				When $p=1$, $\text{DJ}=1+1=2$.  This is an important limit: the limiting dispersity for a step-growth polymerization is 2.
			
			\end{solution}
			
			
			
		\question Can the dispersity of a polymer produced by a step-growth polymerization ever be greater than 2?  Briefly defend your answer in 1-2 complete sentences.
		
			\begin{solution}[1.5in]
			
				Following the argument presented in this exercise, no, the dispersity of a polymer produced by step-growth polymerization can never be greater than 2, because $p$ can never be greater than 1. 
				
				Note for instructors: practically speaking, there are certain conditions that can generate dispersities greater than 2 (for example, when the reaction mixture contains multifunctional monomers that induce chain branching, or when the monomers are added in several batches - see DOI:10.1016/0032-3861(92)90340-3), but for the purposes of this activity, students should learn that the ideal limiting dispersity for step-growth polymerizations is two.
			
			\end{solution}
			
			
\end{ctqs}

\begin{exercises}

		\exercise Suppose you synthesized a polymer by step-growth polymerization and found that it had a dispersity of 1.86.
		
			\begin{enumerate}
				\item What must the extent of reaction have been in this polymerization?
		
					\begin{solution}
					\instructordisplay{
						\begin{equation*}
							p = \text{\DJ}-1 = 1.86-1 = 0.86
						\end{equation*}
					}
					\end{solution}
					
				\item What would you expect the number-average degree of polymerization of this polymer to be?
		
					\begin{solution}
					\instructordisplay{
						\begin{equation*}
							N_n = \frac{1}{1-p} = \frac{1}{1-0.86} = 7.1
						\end{equation*}
					}
					\end{solution}
			\end{enumerate}
			
		\exercise Show that the summation expression for $M_n$ given in Model \ref{\labelbase:mdl:MwMn} simplifies to the expected result by doing the following: \label{labelbase:exc:Mn}
		
			\begin{enumerate}
				\item First, show that the summation expression for $M_n$ given in Model \ref{\labelbase:mdl:MwMn} can be rewritten
					\begin{equation*}
						M_n = M_0 \frac{{\sum_i i p^{i-1}}}{\frac{1}{p}\sum_i p^i}
					\end{equation*}
					
					\begin{solution}\instructordisplay{
						In Model \ref{\labelbase:mdl:MwMn}, $M_n$ was written as
						\begin{equation*}
							M_n = M_0\frac{\sum_i p^{i-1}(1-p)^2 i }{\sum_i p^{i-1}(1-p)^2}
						\end{equation*}
						We can simplify this by realizing that any multiplicative terms that do not depend on $i$ can be pulled out of the sum:
						\begin{equation*}
							M_n = M_0\frac{(1-p)^2\sum_i p^{i-1} i }{(1-p)^2\sum_i p^{i-1}} = M_0\frac{\sum_i p^{i-1} i }{\sum_i p^{i-1}}
						\end{equation*}
						Similarly, using $p^{i-1} = p^i/p$,
						\begin{equation*}
							M_n = M_0\frac{\sum_i p^{i-1} i }{\sum_i p^{i-1}} = M_0\frac{\sum_i p^{i-1} i }{\frac{1}{p}\sum_i p^{i}}
						\end{equation*}
						
					}\end{solution}
					
				\item The denominator of this expression is just a geometric series.  Recall that if $p < 1$, then 
		
			\begin{equation*}
				\sum_{i=1}^{\infty} p^i = \frac{p}{1-p}
			\end{equation*}
			
					Substitute this expression into your equation for $M_n$ and simplify.
					
					\begin{solution}\instructordisplay{
						\begin{align*}
							M_n &= M_0\frac{\sum_i p^{i-1} i }{\frac{1}{p}\sum_i p^{i}}\\
								&= M_0\frac{\sum_i p^{i-1} i }{\frac{1}{p}\frac{p}{1-p}}\\
								&= M_0 (1-p)\sum_i p^{i-1} i 
						\end{align*}
						
					}\end{solution}
			
				\item The remaining sum can be calculated by differentiating both sides of the equation for $\sum_i p^i$.  Carry out this differentiation to show that
						\begin{equation*}
							\sum_{i=1}^{\infty} ip^{i-1} = \frac{1}{(1-p)^2}
						\end{equation*}
				
					\begin{solution}\instructordisplay{
						Left-hand side:
						\begin{align*}
							\frac{d}{dp} \sum_{i=1}^{\infty} p^i &= \sum_{i=1}^{\infty} \frac{d}{dp} p^i\\
							&= \sum_{i=1}^{\infty} ip^{i-1}
						\end{align*}
						Note that the derivative can move through the sum since the derivative depends only on $p$, not $i$.
						
						Right-hand side:
						\begin{align*}
							\frac{d}{dp} \frac{p}{1-p} &= \frac{d}{dp} p(1-p)^{-1}\\
							&= p\frac{d}{dp}(1-p)^{-1} + (1-p)^{-1}\frac{d}{dp} p\\
							&= p(1-p)^{-2} + (1-p)^{-1}\\
							&= \frac{1}{1-p}\left(\frac{p}{1-p} + 1\right)\\
							&= \frac{1}{1-p}\left(\frac{p + 1 - p}{1-p}\right)\\
							&= \frac{1}{1-p}\left(\frac{1}{1-p}\right)\\
							&= \frac{1}{(1-p)^2}
						\end{align*}
						
						Setting them equal, we obtain:
						\begin{equation*}
							\sum_{i=1}^{\infty} ip^{i-1} = \frac{1}{(1-p)^2}
						\end{equation*}
						
					}\end{solution}
					
				\item Finally, substitute this expression into $M_n$ and show that you obtain the expected solution.
				
				
					\begin{solution}\instructordisplay{
						\begin{align*}
							M_n &= M_0 (1-p)\sum_i p^{i-1} i \\
								&= M_0 (1-p)\frac{1}{(1-p)^2} \\
								&= M_0 \frac{1}{1-p}
						\end{align*}
						as expected.
					}\end{solution}
				
			\end{enumerate}
			
\end{exercises}
	
\end{activity}