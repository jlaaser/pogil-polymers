%%%%%%%%%%%%%%%%%%%%%%%%%%%%%%%%%%%%%%%%%
%
% (c) 2018 by Jennifer Laaser
%
% This work is licensed under the Creative Commons Attribution-NonCommercial-ShareAlike 4.0 International License. To view a copy of this license, visit http://creativecommons.org/licenses/by-nc-sa/4.0/ or send a letter to Creative Commons, PO Box 1866, Mountain View, CA 94042, USA.
%
% The current source for these materials is accessible on Github: https://github.com/jlaaser/pogil-polymers
%
%%%%%%%%%%%%%%%%%%%%%%%%%%%%%%%%%%%%%%%%%

\renewcommand{\figpath}{content/polymchem/stepgrowth/condensation-equilibria/figs}

\begin{activity}[Condensation Polymerizations]

\begin{instructornotes}

	This activity introduces students to the role of chemical equilibria in condensation polymerizations.
	
	After completing this activity, students will be able to:
			\begin{enumerate}
				\item Explain how chemical equilibrium affects the extent of reaction in condensation polymerizations
				\item Explain how reaction conditions can be manipulated to produce high molecular-weight polymers
				\item Apply these concepts to the analysis of commercially-important polymer syntheses
			\end{enumerate}
	
			
	\subsection*{Activity summary:}
	\begin{itemize}
		\item \textbf{Activity type:} Learning Cycle
		\item \textbf{Content goals:} Equilibrium effects in condensation polymerizations
		\item \textbf{Process goals:} %https://pogil.org/uploads/attachments/cj54b5yts006cklx4hh758htf-process-skills-official-pogil-list-2015-original.pdf
			written communication, critical thinking, information processing
		\item \textbf{Duration:} TBD %approx. 45 minutes without class discussion
		\item \textbf{Instructor preparation required:} none beyond knowledge of relevant content
		\item \textbf{Related textbook chapters:}
			\begin{itemize}
				\item \emph{Polymer Chemistry} (Hiemenz \& Lodge): section X.X
			\end{itemize}
	\end{itemize}

\end{instructornotes}

	%\textbf{Focus question:} Put a central question for the students to consider through this exercise here.

\begin{model}[Chemistries of Step-Growth Polymerizations]

	(examples of step-growth polymerizations used to produce commercially-important polymers)

\end{model}


\begin{ctqs}

	\question (analysis)
		
\end{ctqs}

\begin{infobox}

(Definition of condensation polymerizations)

\end{infobox}

\begin{ctqs}
		
		\question (identifying polymerizations from model 1 as condensation vs. not)
		
\end{ctqs}
	

\begin{model}[Equilibrium in Condensation Polymerizations]

(example using esters)

\end{model}

\begin{ctqs}
		\question (analysis)
			
\end{ctqs}
	
\begin{infobox}

(reminder of Le Chatelier's principle)

\end{infobox}
	
\begin{ctqs}
		\question (analysis)
		
\end{ctqs}

\begin{model}[Synthesis of Polyethylene Terephthalate]

(Synthetic scheme for PET)

\end{model}

\begin{ctqs}

		\question (analysis)
			
\end{ctqs}
	


\begin{exercises}

		\exercise (extension to other polymer systems?)
\end{exercises}
	
\end{activity}