%%%%%%%%%%%%%%%%%%%%%%%%%%%%%%%%%%%%%%%%%
%
% (c) 2018 by Jennifer Laaser
%
% This work is licensed under the Creative Commons Attribution-NonCommercial-ShareAlike 4.0 International License. To view a copy of this license, visit http://creativecommons.org/licenses/by-nc-sa/4.0/ or send a letter to Creative Commons, PO Box 1866, Mountain View, CA 94042, USA.
%
% The current source for these materials is accessible on Github: https://github.com/jlaaser/pogil-polymers
%
%%%%%%%%%%%%%%%%%%%%%%%%%%%%%%%%%%%%%%%%%

\renewcommand{\figpath}{content/polymchem/stepgrowth/condensation-equilibria/figs}
\renewcommand{\labelbase}{condequilib}

\begin{activity}[Equilibrium in Condensation Polymerizations]

\begin{instructornotes}

	This activity introduces students to the role of chemical equilibria in condensation polymerizations.
	
	After completing this activity, students will be able to:
			\begin{enumerate}
				\item Explain how chemical equilibrium affects the extent of reaction in condensation polymerizations
				\item Explain how reaction conditions can be manipulated to produce high molecular-weight polymers
				\item Apply these concepts to the analysis of commercially-important polymer syntheses
			\end{enumerate}
	
			
	\subsection*{Activity summary:}
	\begin{itemize}
		\item \textbf{Activity type:} Learning Cycle
		\item \textbf{Content goals:} Equilibrium effects in condensation polymerizations
		\item \textbf{Process goals:} %https://pogil.org/uploads/attachments/cj54b5yts006cklx4hh758htf-process-skills-official-pogil-list-2015-original.pdf
			written communication, critical thinking, information processing
		\item \textbf{Duration:} TBD %approx. 45 minutes without class discussion
		\item \textbf{Instructor preparation required:} none beyond knowledge of relevant content
		\item \textbf{Related textbook chapters:}
			\begin{itemize}
				\item \emph{Polymer Chemistry} (Hiemenz \& Lodge): Not covered in detail, but some information in sections 2.5 and 2.6
			\end{itemize}
	\end{itemize}

\end{instructornotes}

	%\textbf{Focus question:} Put a central question for the students to consider through this exercise here.

\begin{model}[Equilibria in Condensation Reactions]

	So far, we have written all reactions as unidirectional, proceeding only from \emph{reactants} to \emph{products}.
	However, in reality, the reactions used to produce polymers by step-growth polymerization are typically reversible, and proceed under equilibrium conditions.
	
	This issue is particularly important for condensation reactions, which produce a small-molecule byproduct.
	Consider the reaction of an ``A'' functional group with a ``B'' functional group to produce an ``ab'' bond:
	
		%\centerline{\includegraphics[width=0.5\textwidth]{\figpath/model1_.pdf}}
	
	Here, we have included the small molecule byproduct, ``SM'' on the right side of the reaction.
	We have also written the reaction arrow as a double arrow ($\leftrightharpoons$) to indicate that the reaction is reversible.

\end{model}


\begin{ctqs}

	\question \label{\labelbase:ctq:Keq} Write an expression for the equilibrium constant for this reaction, $K_{eq}$, in terms of the concentrations of A groups ([A]), B groups ([B]), ab bonds ([ab]), and released small molecules ([SM]):
	
		\begin{solution}[1.5in]
			\begin{equation*}
				K_{eq} = \frac{[ab][SM]}{[A][B]}
			\end{equation*}
		\end{solution}
	
	\question \label{\labelbase:ctq:ICE} Suppose we start with $v_A^0$ A groups.  If the reaction is stoichiometrically balanced, that means we also start with $v_A^0$ B groups.
	
		Using this information, complete the following ICE table:
		\begin{center}
			\renewcommand{\arraystretch}{4}
			\begin{tabular}{|c|c|c|c|c|}
				\hline
				~ & ~~~~~~~\textbf{A}~~~~~~~ & ~~~~~~~\textbf{B}~~~~~~~ & ~~~~~~~\textbf{ab}~~~~~~~ & ~~~~~~~\textbf{SM}~~~~~~~\\\hline
				\textbf{Initial} & $v_A^0$ & $v_A^0$ & 0 & 0 \\\hline
				\textbf{Change} & $-x$ & \answer{$-x$} & \answer{$+x$} & \answer{$+x$} \\\hline
				\textbf{Equilibrium} & \answer{$v_A^0 - x$} & $v_A^0-x$ & \answer{$x$} & \answer{$x$} \\\hline
			\end{tabular}
		\end{center}
		
	\question Plug your values from the equilibrium line of this table into your expression from question \ref{\labelbase:ctq:Keq} to find an expression for $K_{eq}$ in terms of $v_A^0$ and $x$.
	
		%\emph{Note: your expression from question \ref{\labelbase:ctq:Keq} is written in terms of concentrations, while the values in the table in question \ref{\labelbase:ctq:ICE} are written in terms of numbers of molecules. However, }
	
		\begin{solution}[1.5in]
			\begin{equation*}
				K_{eq} = \frac{x^2}{(v_A^0-x)^2}
			\end{equation*}
		\end{solution}
		
\end{ctqs}

\begin{infobox}

	CONNECT TO POLYMERS/DEGREE OF POLYMERIZATION
	
	When the extent of reaction is equal to $p$, the number of $A$ groups that we have reacted is $pV_A^0$, so $x=pV_A^0$. 

\end{infobox}

\begin{ctqs}
	
	\question  Using this information, rewrite $K_{eq}$ only in terms of $p$.
		
		\begin{solution}[1.5in]
			\begin{equation*}
				K_{eq} = \frac{(pV_A^0)^2}{(v_A^0 - pv_A^0)^2} = \frac{p^2}{(1-p)^2}
			\end{equation*}
		\end{solution}
	
	
	\question Finally, recalling that the number-average degree of polymerization is related to $p$ by $N_n = \frac{1}{1-p}$, show that 
		\begin{equation*}
			N_n = \sqrt{K_{eq}} + 1
		\end{equation*}
		
		\emph{Hint: you will probably find it easiest to just plug your expression for $K_{eq}$ into the right-hand side of this equation and simplify.}
		
		\begin{solution}[1.5in]
			$\sqrt{K_{eq}} = \frac{p}{1-p}$ so $\sqrt{K_{eq}}+1 = \frac{p}{1-p} + 1 = \frac{p}{1-p} + \frac{1-p}{1-p} = \frac{p + 1 -p}{1-p} = \frac{1}{1-p} = N_n$
		\end{solution}
		
\end{ctqs}
	

\begin{model}[Equilibrium in Condensation Polymerizations]

Polyesters and polyamides are two important classes of polymers formed by condensation polymerization.

Shown below are the bond-forming reactions and equilibrium constants for typical esterification and amidation reactions:

(FIGURE)

\end{model}

\begin{ctqs}
		\question Calculate the expected value of $N_n$ for a polyester synthesized under equilibrium conditions.
		
		\question Calculate the expected value of $N_n$ for a polyamide synthesized under equilibrium conditions.
		
		\question Commercially applications of polyesters and polyamides typically require degrees of polymerization of 100 or more.  Can these polymers be produced under equilibrium conditions?  Why or why not?
			
\end{ctqs}
	
\begin{infobox}

(reminder of Le Chatelier's principle)

\end{infobox}
	
\begin{ctqs}
		\question (analysis)
		
\end{ctqs}

\begin{model}[Synthesis of Polyethylene Terephthalate]

(Synthetic scheme for PET)

\end{model}

\begin{ctqs}

		\question (analysis)
			
\end{ctqs}
	


\begin{exercises}

		\exercise (extension to other polymer systems?)
\end{exercises}
	
\end{activity}