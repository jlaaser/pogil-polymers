\section{For Students}

As you'll probably notice, this book looks a little different from a standard textbook.  That's because it is intended not to directly \emph{tell} you what you should know about polymers, but instead to \emph{guide you through} (with help from your instructor) \emph{figuring it out yourself}.  This approach will feel very different if you are used to traditional lecture courses, but we hope you will find that taking an active role in your own learning will help you develop deeper understanding of - and more confidence in - the course material. % Learning is not a spectator sport - get in there and get your hands dirty!

Each activity in this book is broken into two to four Models, which present key reactions, data, or other information, followed by critical thinking questions that will guide you through exploring that information and developing your own understanding of the core concepts behind the data.
Your instructor will likely ask you to work through these activities in small groups, with class discussions interspersed throughout to emphasize key points.  We encourage you to engage actively with your group and your class throughout - this is a great chance to learn from your peers, to practice explaining difficult concepts, and to build teamwork skills that will help you succeed in your future career.

Along the way, we hope you enjoy learning about polymer science - this is a really fun field, and if it captures your interest, there are a lot of places you can go with it!