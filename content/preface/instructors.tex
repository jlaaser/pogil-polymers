\section{For Instructors}

\ifbool{instructorguide}{% this section will show if we are compiling the instructor guide

	\color{red}

	% This is a messy way to do this - should figure out a way to get it into the class

	Thanks for your interest in bringing guided inquiry into your course in polymer chemistry or polymer physics!  As noted in the general preface, above, this book is the result of a project by the author to develop guided inquiry activities for her polymer science course at the University of Pittsburgh, and we hope that they will be a useful resource for incorporating guided inquiry activities in your courses, too.

While these activities can be implemented using a variety of classroom approaches, they have been designed using the framework of Process Oriented Guided Inquiry Learning (POGIL), in which students work in groups to explore models and then use the information in those models to invent concepts \& apply their knowledge to new situations. If you are new to this approach, \href{https://www.pogil.org/about-pogil/what-is-pogil}{The POGIL Project} offers a number of resources that are a great starting point for implementing this instructional method.  Please note, however, that this set of activities are not affiliated with or endorsed by The POGIL Project, and any specific inquiries regarding implementation of the content in this book should be directed to \href{mailto:j.laaser@pitt.edu}{its author}.

All activities included in this version of the book have been classroom-tested in the author's course and revised to address common issues encountered by the students.  A number of additional activities are under active development; as drafts are completed, these activities will be made available through the GitHub repository for this project (\url{https://github.com/jlaaser/pogil-polymers} - see branches labeled ``WIP''), and can also be obtained by emailing the author at \href{mailto:j.laaser@pitt.edu}{j.laaser@pitt.edu}.  Full instructor's solutions for all activities in this collection are also available in the GitHub repository (see README file for compilation instructions) or by emailing the author.

Finally, to enable you to make the best use of this material, all of the activities in this collection are licensed under the \href{http://creativecommons.org/licenses/by-nc-sa/4.0/}{Creative Commons Attribution-NonCommercial-ShareAlike 4.0 International License}.  Under this license, you are welcome to modify or adapt any of the activities to meet the needs of your class, as long as the attribution to the author remains attached.  We also welcome suggestions for additions or revisions to improve the current set of activities - please don't hesitate to reach out via email if you have ideas you would like to discuss!

	\color{black}

}{% this section will show if we are compiling the student version

	Thanks for your interest in \thetitle{!} You are viewing the student version of this book.  The instructor's guide, which includes an extended preface and implementation notes for all activities, can be obtained from the \href{https://github.com/jlaaser/pogil-polymers}{GitHub repository} (see README file for compilation instructions) or by \href{mailto:j.laaser@pitt.edu}{emailing the author}.  A full solutions guide is also available; please contact the author for more information.
	
	This collection is a work in progress, and more activities are under active development.  If you are interested in an activity not included in the current collection, please contact the author - it is possible that a draft version will be available. 
	
	We hope you enjoy bringing guided inquiry into your polymer science class!

}

